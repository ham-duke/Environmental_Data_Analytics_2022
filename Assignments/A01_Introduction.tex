% Options for packages loaded elsewhere
\PassOptionsToPackage{unicode}{hyperref}
\PassOptionsToPackage{hyphens}{url}
%
\documentclass[
]{article}
\title{Assignment 1: Introduction}
\author{Halina Malinowski}
\date{}

\usepackage{amsmath,amssymb}
\usepackage{lmodern}
\usepackage{iftex}
\ifPDFTeX
  \usepackage[T1]{fontenc}
  \usepackage[utf8]{inputenc}
  \usepackage{textcomp} % provide euro and other symbols
\else % if luatex or xetex
  \usepackage{unicode-math}
  \defaultfontfeatures{Scale=MatchLowercase}
  \defaultfontfeatures[\rmfamily]{Ligatures=TeX,Scale=1}
\fi
% Use upquote if available, for straight quotes in verbatim environments
\IfFileExists{upquote.sty}{\usepackage{upquote}}{}
\IfFileExists{microtype.sty}{% use microtype if available
  \usepackage[]{microtype}
  \UseMicrotypeSet[protrusion]{basicmath} % disable protrusion for tt fonts
}{}
\makeatletter
\@ifundefined{KOMAClassName}{% if non-KOMA class
  \IfFileExists{parskip.sty}{%
    \usepackage{parskip}
  }{% else
    \setlength{\parindent}{0pt}
    \setlength{\parskip}{6pt plus 2pt minus 1pt}}
}{% if KOMA class
  \KOMAoptions{parskip=half}}
\makeatother
\usepackage{xcolor}
\IfFileExists{xurl.sty}{\usepackage{xurl}}{} % add URL line breaks if available
\IfFileExists{bookmark.sty}{\usepackage{bookmark}}{\usepackage{hyperref}}
\hypersetup{
  pdftitle={Assignment 1: Introduction},
  pdfauthor={Halina Malinowski},
  hidelinks,
  pdfcreator={LaTeX via pandoc}}
\urlstyle{same} % disable monospaced font for URLs
\usepackage[margin=2.54cm]{geometry}
\usepackage{graphicx}
\makeatletter
\def\maxwidth{\ifdim\Gin@nat@width>\linewidth\linewidth\else\Gin@nat@width\fi}
\def\maxheight{\ifdim\Gin@nat@height>\textheight\textheight\else\Gin@nat@height\fi}
\makeatother
% Scale images if necessary, so that they will not overflow the page
% margins by default, and it is still possible to overwrite the defaults
% using explicit options in \includegraphics[width, height, ...]{}
\setkeys{Gin}{width=\maxwidth,height=\maxheight,keepaspectratio}
% Set default figure placement to htbp
\makeatletter
\def\fps@figure{htbp}
\makeatother
\setlength{\emergencystretch}{3em} % prevent overfull lines
\providecommand{\tightlist}{%
  \setlength{\itemsep}{0pt}\setlength{\parskip}{0pt}}
\setcounter{secnumdepth}{-\maxdimen} % remove section numbering
\ifLuaTeX
  \usepackage{selnolig}  % disable illegal ligatures
\fi

\begin{document}
\maketitle

\hypertarget{overview}{%
\subsection{OVERVIEW}\label{overview}}

This exercise accompanies the introductory material in Environmental
Data Analytics.

\hypertarget{directions}{%
\subsection{Directions}\label{directions}}

\begin{enumerate}
\def\labelenumi{\arabic{enumi}.}
\tightlist
\item
  Change ``Student Name'' on line 3 (above) with your name.
\item
  Work through the steps, \textbf{creating code and output} that fulfill
  each instruction.
\item
  Be sure to \textbf{answer the questions} in this assignment document.
\item
  When you have completed the assignment, \textbf{Knit} the text and
  code into a single PDF file.
\item
  After Knitting, submit the completed exercise (PDF file) to the
  dropbox in Sakai. Add your last name into the file name (e.g.,
  ``Lima\_A01\_Introduction.Rmd'') prior to submission.
\end{enumerate}

The completed exercise is due on \textless\textgreater.

\hypertarget{discussion-questions}{%
\subsection{1) Discussion Questions}\label{discussion-questions}}

\begin{enumerate}
\def\labelenumi{\arabic{enumi}.}
\tightlist
\item
  What are your previous experiences with data analytics, R, and Git?
  Include both formal and informal training.
\end{enumerate}

\begin{quote}
Answer: I have some experience with data analytics, R, and Git. However,
my level with each is a bit varied. Fall of 2020, I took ENV 710,
Applied Data Analysis for Environmental Sciences. This course covered
several statistics and some data processing in R. At the end of the
semester I completed a project on data I had prevoiously collected and
worked through the process of data wrangling and running some simpled
statistical code in R. As part of my lab I have also been involved with
an on-going project on above ground biomass in Gabon. Here, I have used
R to clean data and have done a little mapping work in R. Additionally,
I am part of a BASS Connections project where I have worked in R to run
some statistics on biodiversity and above ground carbon from our labs
datasets. I have also had a little bit more exposure to R in a few
courses including Landscape Analysis and Management and Food Web Theory.
At the very end of last semester, my lab decided to move all of our code
for our above ground biomass project on to Github. Therefore, I
installed Git on my computer and have been using it when working on our
project. However, Git is still very new to me. I do have some background
in R at this point, but I still feel very uncertain and feel that I need
a lot more practice to develop this skill further. At the moment coding
does not come naturally to me and I need to do a lot of googling to work
my way through various projects. I certainly would like to get more
comfortable with R, understanding the ``syntax'' of the language and
error messages, and feel more confident in my abilities to tactle data
and analysis in R.
\end{quote}

\begin{enumerate}
\def\labelenumi{\arabic{enumi}.}
\setcounter{enumi}{1}
\tightlist
\item
  Are there any components of the course about which you feel confident?
\end{enumerate}

\begin{quote}
Answer: I feel that the introduction to the course and maybe the first
week or so will be a bit easier for me and a good review. However, I do
not feel particularly confident in my R skills, but I am really looking
forward to sharpening these skills. Therefore, I am confident in my
ability to work hard at this course and put in the necessary time to
solidify these skills so that I can advance my career. I believe I am
also becoming a bit better at trouble shooting in R so that will
hopefully be advantageous. I also understand some of the basic syntax in
R and how to manuever through RStudio. I feel confident in importing
datasets and some basic data exploration as well as some visualization
making basic plots and using ggplot. I really see the usefulness in
being able to write, comprehend, and analyze R code so I am eager to
jump into this course!
\end{quote}

\begin{enumerate}
\def\labelenumi{\arabic{enumi}.}
\setcounter{enumi}{2}
\tightlist
\item
  Are there any components of the course about which you feel
  apprehensive?
\end{enumerate}

\begin{quote}
Answer: I am certainly a bit apprehensive about this course especially
as we get further into the semester. I have used R Markdown before, but
I've run into problems several times. I'm certain that I can
troubleshoot this, but I am certainly apprehensive about Markdown. I
have also never used R tidy and have really only worked with base R so I
am eager to learn about tidy, but also a bit worried about the different
methods for coding using tidy. I think some of the larer weeks in the
course might be a bit tricky for me such as time series analysisy and
various trend analysis and observations. Furthermore, I am very excited
to learn about R shiny. One of my labmates has used R shiny quite a bit
in there research and visualization and I would love to learn how to use
it so I may apply it to my own work. That being said I am a bit
apprehensive about it because I've never done anything with R shiny or R
dashboards and really do not have any background in what goes into it.
The python from an R perspective lesson also looks extremely useful
especially given some of my work in GIS, but I have never coded in
python so this will likely be a challenge for me as well. Either way I'm
enthusiastic to be starting this course and think it will be a great
learning experience and very beneficial for my own work.
\end{quote}

\hypertarget{github}{%
\subsection{2) GitHub}\label{github}}

Provide a link below to your forked course repository in GitHub. Make
sure you have pulled all recent changes from the course repository and
that you have updated your course README file.

\begin{quote}
Answer:
\url{https://github.com/ham-duke/Environmental_Data_Analytics_2022.git}
\end{quote}

\end{document}
